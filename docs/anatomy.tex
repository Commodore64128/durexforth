\chapter{Word Anatomy}

\section{Inspecting a Word}

Let us define a word and see what it gets compiled to.

\begin{verbatim}
: bg d020 c! ;
\end{verbatim}

When the word is defined, you can get its start address by \texttt{loc bg}, and the contents of bg can be dumped using \texttt{loc bg dump}. Try it, and you will get output like the following:

\begin{alltt}
4c38  ed 4b 02 42 47 20 cf 0e .k.bg ..
4c40  20 d0 20 49 0a 60 ff ff  . i....
4c48  ff ff ff ff ff ff ff ff ........
4c50  ...
\end{alltt}

Here, we can see that the "bg" word is 14 bytes long and starts at address \$4c38. It contains two parts: Header and code.

\section{Header}

\begin{alltt}
4c38  \textcolor{red}{ed 4b 02 42 47} 20 cf 0e \textcolor{red}{.k.bg} ..
4c40  20 d0 20 49 0a 60 ff ff  . i....
\end{alltt}

The first two bytes contain a back-pointer to the previous word, starting at \$4bed. The next byte, "02", is the length of "bg" name string. After that, the string "bg" follows. (42 = 'b', 47 = 'g')

The name length byte is also used to store special attributes of the word. Bit 7 is "immediate" flag, which means that the word should execute immediately instead of being compiled into word definitions. ("(" is such an example of an immediate word that does not get compiled.) Bit 6 is "hidden" flag, which makes a word unfindable. Bit 5 is the "no-tail-call-elimination" flag, which makes sure that tail call elimination (the practice of replacing jsr/rts with jmp) is not performed if this word is the jsr target. Since bg does not have these flags set, bits 7-5 are all clear.

\section{Code}

\begin{alltt}
4c38  ed 4b 02 42 47 \textcolor{red}{20 cf 0e} .k.bg\textcolor{red}{ ..}
4c40  \textcolor{red}{20 d0 20 49 0a 60 ff ff  . i..}..
\end{alltt}

The code section contain pure 6502 machine code.

\begin{description}
\item[20 cf 0e ( jsr \$ecf )] \$ecf is the adress of the \texttt{lit} code. \texttt{lit} copies the two following bytes to parameter stack.
\item[20 d0 ( \$d020 )] The parameter to the \texttt{lit} word. When executed, \texttt{lit} will add \$d020 to the parameter stack.
\item[20 49 0a ( jsr \$a49 )] \$a49 is the address of the \texttt{c!} code.
\item[60 ( rts )] Returns to the caller.
\end{description}
